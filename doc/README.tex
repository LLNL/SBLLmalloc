/*!

@mainpage SBLLmalloc
@author Susmit Biswas

Memory size has long limited large-scale applications on high-performance 
computing (HPC) systems. Increasing core counts per chip and power density 
constraints, which limit the number of DIMMs per node, have exacerbated this 
problem. Mechanisms to manage memory usage ---preferably transparently--- more 
efficiently could increase effective DRAM capacity and, thus, the benefit 
of multicore nodes for HPC systems.

MPI application processes often exhibit significant data similarity. 
These data regions occupy multiple physical locations within a multicore 
node and thus offer a potential savings in memory capacity. These data, 
primarily residing in heap, are quite dynamic, which makes them difficult 
to manage statically.

SBLLmalloc is a memory allocation library that
automatically identifies the replicated memory blocks and merges them into
a single copy. SBLLmalloc does not require application or OS changes since
we implement it as a user-level library that can be linked at runtime.
Overall, we find that SBLLmalloc reduces the memory
footprint of a wide range of MPI applications by \f$37.03\%\f$ on average
and up to \f$60.87\%\f$. Further, \SBLLmalloc supports problem sizes for 
AMG and IRS over \f$18.5\%\f$ and \f$21.6\%\f$ larger than using standard memory 
management techniques, thus significantly increasing effective system size.

In the following, the usage of the library is described. In the \e run
directory you will find a script called submitjob.sh.  Modify it to your need.
Examples of using this script is shown at the end of this section.

This file describes the usage of the heap merging library. In order to use
merge capability, you need to set some environment variables which triggers
merge operations. If the original commandline is 

	srun -nx -Ny <commandline>, 
	
you will need to change it to the following command.

\verbatim
MPIRUN="srun -nx -Ny" TH=<threshold value> ENV_VAR1=<value1> \
	[ENV_VAR2=<value2> ...] $TOPDIR/run/submitjob_v2.sh <commandline>
\endverbatim

At the end of the section you will find a concrete example.
	
If address space layout randomization (ASLR) is enabled, for x86_64 machines
run the program with setarch x86_64 --3gb -R.  For i*86 based servers, use i386
instead of x86_64. In most of the HPC clusters ASLR is disabled, so you may
need to use \c setarch at all. Check the file \c
/proc/sys/kernel/randomize_va_space to see if the value set 0 to disable ASLR.

In order to set parameters in the library you need to set some environment
variables which are listed in Table\latexonly \ref{tab:env-vars}\endlatexonly

\latexonly
\begin{table}[!tb]
\centering
\begin{tabular}{|c|c|c|}
\hline \hline
\textbf{Name} & \textbf{Default} & \textbf{Description} \\ \hline
PROFILE\_MODE & 1 & profiling mode? \\
& & 0: no profiling \\
& & 1: create profiles \\
& & 2: use profile for merging (EXPERIMENTAL) \\ \hline
MERGE\_METRIC & 1 & merge metric? \\
& & 0: disabled \\
& & 1: alloc\_frequency \\
& & 2: threshold (Recommended)\\
& & 3: buffered (Experimental)\\ \hline
MALLOC\_MERGE\_FREQ & 1000 & frequency for frequency based merge \\ \hline
MIN\_MEM\_TH & 10 & threshold for threshold based merge \\ \hline
ENABLE\_BACKTRACE & 0 & enable backtrace?\\
& & 1: enabled\\
& & 0: disabled\\
& & Used for finding the source location that\\
& & allocated the merged page \\  \hline
SEM\_KEY & 1234 & semaphore key \\ \hline
NOT\_MPI\_APP & 0 & define 1 if this does not call MPI\_Init(). \\
& & You need to modify the code. Please read the TODO list.\\ \hline
\end{tabular}
\label{tab:env-vars}
\caption{Environment Variables}
\end{table}
\endlatexonly

Example use:
\verbatim
bash$ cat run.amg.sh
#!/bin/bash
COMMANDLINE="./amg2006 -P 2 2 2 -n 80 80 80 -r 20 20 20 -27pt"
MPIRUN="srun -n8 -N1" TH=200 MERGE_METHOD=2 ENABLE_BACKTRACE=0 PROF=0 \ 
	~/LLNL_WORK/ptmalloc/run/submitjob_v2.sh $COMMANDLINE
bash$ ./run.amg.sh
salloc: Granted job allocation 1006112
  Laplace type problem with a 27-point stencil 
  (nx_global, ny_global, nz_global) = (160, 160, 160)
  (Px, Py, Pz) = (2, 2, 2)
   ...
\endverbatim

If you get a fault due to mmap cap, issue the following command to change the
default max map count to 512K. In default system configuration it is set as
64K. Check the value with  the following command.

\em sysctl \em vm.max_map_count

To change the limit please issue the following command.

\c sudo \c sysctl \c vm.max_map_count=$((512*1024))

*/	
